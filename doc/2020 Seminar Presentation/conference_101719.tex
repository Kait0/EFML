\documentclass[conference]{IEEEtran}
\usepackage{cite}
\usepackage{amsmath,amssymb,amsfonts}
\usepackage{algorithmic}
\usepackage{graphicx}

\usepackage{textcomp}
\usepackage{xcolor}
\def\BibTeX{{\rm B\kern-.05em{\sc i\kern-.025em b}\kern-.08em
    T\kern-.1667em\lower.7ex\hbox{E}\kern-.125emX}}


    
\begin{document}

\title{Preliminary Title}

\author{
	\IEEEauthorblockN{Bernhard Jaeger}
	\IEEEauthorblockA{
		\textit{University of Tübingen}\\
		Tübingen, Germany \\
		bernhard.jaeger@student.uni-tuebingen.de
	}
}

\maketitle

\begin{abstract}
This document is a model and instructions for \LaTeX.
This and the IEEEtran.cls file define the components of your paper [title, text, heads, etc.]. *CRITICAL: Do Not Use Symbols, Special Characters, Footnotes, 
or Math in Paper Title or Abstract.
\end{abstract}

\begin{IEEEkeywords}
component, formatting, style, styling, insert
\end{IEEEkeywords}

\section{Introduction}
This document is a model and instructions for \LaTeX.
Please observe the conference page limits. 

\section{SHAP values}

\section{Decision and Hoeffding Trees}

\subsection{Abbreviations and Acronyms}\label{AA}
Define abbreviations and acronyms the first time they are used in the text, 
even after they have been defined in the abstract. Abbreviations such as 
IEEE, SI, MKS, CGS, ac, dc, and rms do not have to be defined. Do not use 
abbreviations in the title or heads unless they are unavoidable.

\subsection{Equations}
Number equations consecutively. To make your 
equations more compact, you may use the solidus (~/~), the exp function, or 
appropriate exponents. Italicize Roman symbols for quantities and variables, 
but not Greek symbols. Use a long dash rather than a hyphen for a minus 
sign. Punctuate equations with commas or periods when they are part of a 
sentence, as in:
\begin{equation}
a+b=\gamma\label{eq}
\end{equation}

\subsection{\LaTeX-Specific Advice}

Please use ``soft'' (e.g., \verb|\eqref{Eq}|) cross references instead
of ``hard'' references (e.g., \verb|(1)|). That will make it possible
to combine sections, add equations, or change the order of figures or
citations without having to go through the file line by line.

\subsection{Figures and Tables}
\paragraph{Positioning Figures and Tables} Place figures and tables at the top and 
bottom of columns. Avoid placing them in the middle of columns. Large 
figures and tables may span across both columns. Figure captions should be 
below the figures; table heads should appear above the tables. Insert 
figures and tables after they are cited in the text. Use the abbreviation 
``Fig.~\ref{fig}'', even at the beginning of a sentence.

\begin{table}[htbp]
\caption{Table Type Styles}
\begin{center}
\begin{tabular}{|c|c|c|c|}
\hline
\textbf{Table}&\multicolumn{3}{|c|}{\textbf{Table Column Head}} \\
\cline{2-4} 
\textbf{Head} & \textbf{\textit{Table column subhead}}& \textbf{\textit{Subhead}}& \textbf{\textit{Subhead}} \\
\hline
copy& More table copy$^{\mathrm{a}}$& &  \\
\hline
\multicolumn{4}{l}{$^{\mathrm{a}}$Sample of a Table footnote.}
\end{tabular}
\label{tab1}
\end{center}
\end{table}

\begin{figure}[htbp]
\centerline{\includegraphics{fig1.png}}
\caption{Example of a figure caption.}
\label{fig}
\end{figure}

\section{Methodology}

\section{Evaluation}

\section{Conclusion}

\section{Reproducibility Considerations}

\subsection{Software Libraries:}
The software library versions we used (read out using pip --freeze) are documented in table \ref{softwareVersion}.\\
All experiments were conducted on a single personal computer. The hardware specs can be found in table \ref{hardware}.\\


%TODO this are too many remove the uneccesary once.
\begin{table}[htbp]
\caption{Software Library Versions}
\begin{center}
\begin{tabular}{|c|c|}
	\hline
	\textbf{Software library} & \textbf{Version} \\
	\hline
	Python & 3.7.4\\ \hline
	anaconda-client & 1.7.2 \\ \hline
	anaconda-navigator & 1.9.7 \\ \hline
	anaconda-project & 0.8.3\\ \hline
	conda & 4.7.12\\ \hline
	conda-build & 3.18.9\\ \hline
	conda-package-handling & 1.6.\\ \hline0
	conda-verify & 3.4.2\\ \hline
	ipykernel & 5.1.2\\ \hline
	ipython & 7.8.0\\ \hline
	ipython-genutils & 0.2.0\\ \hline
	jupyter & 1.0.0\\ \hline
	jupyter-client & 5.3.3\\ \hline
	jupyter-console & 6.0.0\\ \hline
	jupyter-core & 4.5.0\\ \hline
	jupyterlab & 1.1.4\\ \hline
	jupyterlab-server & 1.0.6\\ \hline
	matplotlib & 3.1.1\\ \hline
	notebook & 6.0.1\\ \hline
	numpy & 1.16.5\\ \hline
	pandas & 0.25.1\\ \hline
	scikit-image & 0.15.0\\ \hline
	scikit-learn & 0.21.3\\ \hline
	scikit-multiflow & 0.4.1\\ \hline
	scipy & 1.3.1\\ \hline
	shap & 0.35.0\\ \hline
	sklearn & 0.0\\ \hline
	torch & 1.5.0+cu101\\ \hline
	torchvision & 0.6.0+cu101  \\ \hline
	
\end{tabular}
\label{softwareVersion}
\end{center}
\end{table}

\begin{table}[htbp]
\caption{Hardware Specifications:}
\begin{center}
\begin{tabular}{|c|c|}
	\hline
	\textbf{Component} & \textbf{Model} \\
	\hline
	Processor & Intel Core i7-6700 CPU\\ \hline
	RAM & 16 GB Corsair Vengeance LPX DDR4-2400\\ \hline
	GPU & NVIDIA GeForce GTX 980 \\ \hline
	Motherboard & ASUS Z170 Pro Gaming \\ \hline
	Operating system & Microsoft Windows 10 Education N \\ \hline
	Hard Drive & SSD Crucial MX200 \\ \hline
\end{tabular}
\label{hardware}
\end{center}
\end{table}




\section*{Acknowledgment}

The preferred spelling of the word ``acknowledgment'' in America is without 
an ``e'' after the ``g''. Avoid the stilted expression ``one of us (R. B. 
G.) thanks $\ldots$''. Instead, try ``R. B. G. thanks$\ldots$''. Put sponsor 
acknowledgments in the unnumbered footnote on the first page.

\section*{References}

Please number citations consecutively within brackets \cite{b1}.

\begin{thebibliography}{00}
\bibitem{b1} G. Eason, B. Noble, and I. N. Sneddon, ``On certain integrals of Lipschitz-Hankel type involving products of Bessel functions,'' Phil. Trans. Roy. Soc. London, vol. A247, pp. 529--551, April 1955.
\bibitem{b2} J. Clerk Maxwell, A Treatise on Electricity and Magnetism, 3rd ed., vol. 2. Oxford: Clarendon, 1892, pp.68--73.
\bibitem{b3} I. S. Jacobs and C. P. Bean, ``Fine particles, thin films and exchange anisotropy,'' in Magnetism, vol. III, G. T. Rado and H. Suhl, Eds. New York: Academic, 1963, pp. 271--350.
\bibitem{b4} K. Elissa, ``Title of paper if known,'' unpublished.
\bibitem{b5} R. Nicole, ``Title of paper with only first word capitalized,'' J. Name Stand. Abbrev., in press.
\bibitem{b6} Y. Yorozu, M. Hirano, K. Oka, and Y. Tagawa, ``Electron spectroscopy studies on magneto-optical media and plastic substrate interface,'' IEEE Transl. J. Magn. Japan, vol. 2, pp. 740--741, August 1987 [Digests 9th Annual Conf. Magnetics Japan, p. 301, 1982].
\bibitem{b7} M. Young, The Technical Writer's Handbook. Mill Valley, CA: University Science, 1989.
\end{thebibliography}

\end{document}
